\documentclass[a4paper,10pt]{article} 
\usepackage[left=0.8in, right=0.8in, top=0.6in, bottom=0.6in]{geometry} 
\usepackage{enumitem} 
\usepackage{titlesec} 
\usepackage{xcolor} 
\usepackage{fontspec} 
\usepackage{xeCJK} 
\usepackage{graphicx} 
\usepackage{wrapfig} 
\usepackage{tikz}
\usepackage{fontawesome5}

% 颜色定义
\definecolor{mypurple}{RGB}{70, 130, 180} 
\definecolor{mygradient}{RGB}{200, 220, 240}

% 标题格式与梯形设计
\newcommand{\trapezoidtitleA}[1]{
    \vspace{0cm} % 增加前后的空间
    \noindent
    \begin{tikzpicture}[baseline=(current bounding box.south)]
        \fill[mypurple] (0,0) -- (0,0.8) -- (2.8,0.8) -- (2.3,0) -- cycle;
        \node[anchor=west, text width=3.5cm, align=lift, text=white] at (0,0.4) {\Large\bfseries #1};
    \end{tikzpicture}
    \vspace{0.5cm}
    \noindent
}

\newcommand{\trapezoidtitleB}[1]{
    \vspace{0cm} % 增加前后的空间
    \noindent
    \begin{tikzpicture}[baseline=(current bounding box.south)]
        \fill[mypurple] (0,0) -- (0,0.8) -- (2,0.8) -- (1.5,0) -- cycle;
        \node[anchor=west, text width=3.5cm, align=lift, text=white] at (0,0.4) {\Large\bfseries #1};
    \end{tikzpicture}
    \vspace{0.5cm}
    \noindent
}


% 标题格式
\titleformat{\section}{\Large\bfseries\color{white}}{}{0em}{\color{mypurple}\underline} 
\titleformat{\subsection}{\bfseries\color{mypurple}}{}{0em}{}

\setCJKmainfont{Noto Serif CJK SC} 
\setmainfont{Noto Serif} 

\begin{document}
\renewcommand{\baselinestretch}{1.5} % 设置行间距
\begin{center} % 居中环境
    % 个人信息部分
    {\LARGE \textbf{风玲}} \\[0.2cm] % 姓名,大号字体
    女 | 汉族 \\[0.2cm] % 性别与民族
    \faEnvelope \ abc@gmail.com \quad % 邮箱
    \faPhone \ 123456789 \quad % 电话
    \faLinkedin \ www.linkedin.com/in/abc \quad % LinkedIn链接
    \faGithub \ www.github.com/abc % GitHub链接
    \vspace{0.5cm} % 增加垂直间距
    {\color{mypurple}{\rule{1\linewidth}{1pt}}} % 蓝色分割线
\end{center}
\vspace{-1cm} % 增加与下一个部分的间距

\trapezoidtitleA{教育背景}
\begin{wrapfigure}{l}{0.15\textwidth} % 左侧包围图片,宽度为文本框的15%
    \includegraphics[width=0.1\textwidth]{school.png} % 大学徽标,设置宽度
\end{wrapfigure} 
\vspace{0cm} % 增加垂直间距
\textbf{联合国大学}|统招全日制本科|专业:人工智能\\ % 教育机构与专业,粗体显示
2021年9月 -- 2025年6月 \\ % 学习时间段
\begin{itemize}[leftmargin=*] % 列表,左侧无缩进
    \item 主要课程: 数据结构, 机器学习, 自然语言处理, 计算智能, 数据挖掘, 深度神经网络, 模式识别 % 课程列表
\end{itemize}
\vspace{0cm} % 调整与分割线的间距
{\color{mypurple!60!black}{\rule{1\linewidth}{1pt}}} % 画一条分割线,背景颜色稍微变暗的紫色
\vspace{-0.5cm} % 增加与下一个部分的间距

\trapezoidtitleB{技能}
\begin{itemize}[leftmargin=*]
    \item \textcolor{mypurple}{\faPython} \ \textbf{编程语言}: 掌握Python,C++等编程语言,了解Java和MATLAB。
    \item \textcolor{mypurple}{\faGit} \ \textbf{工具与环境}: 熟悉Git版本控制,熟练Linux系统常见发行版本,使用Docker进行环境部署与管理。
    \item \textcolor{mypurple}{\faBrain} \ \textbf{人工智能}: 熟悉TensorFlow、PyTorch、Keras等框架,用于深度学习模型的开发与优化。
    \item \textcolor{mypurple}{\faRobot} \ \textbf{机器学习与深度学习}: 具备监督学习、无监督学习与强化学习的理论基础,知晓如何构建卷积神经网络(CNN)、循环神经网络(RNN)、长短期记忆网络(LSTM)等。
    \item \textcolor{mypurple}{\faChartLine} \ \textbf{数据处理与分析}: 熟练使用NumPy、Pandas等工具进行数据清洗、分析与可视化,能够使用Matplotlib、Seaborn进行数据可视化。
    \item \textcolor{mypurple}{\faKeyboard} \ \textbf{自然语言处理}: 熟悉文本预处理、词向量(Word2Vec)、Transformer与BERT等预训练模型,应用于文本分类、生成及翻译任务。
    \item \textcolor{mypurple}{\faCameraRetro} \ \textbf{计算机视觉}: 掌握OpenCV及深度学习在图像处理、目标检测与图像生成中的应用。
    \item \textcolor{mypurple}{\faDatabase} \ \textbf{数据库}: 熟练操作SQL server、SQLite等数据库。
\end{itemize}

{\color{mypurple}\rule{1\linewidth}{1pt}}

\vspace{-0.1cm}
\trapezoidtitleA{实训经历}
\noindent

\parbox{\linewidth}{
    \textbf{前后端开发} | 成都易腾创想科技有限公司 \\2024年8月 -- 2024年9月 \\
    \begin{itemize}[leftmargin=*]
   	 \item 负责数据库的搭建及优化
  	 \item 设计并实现商品推荐模块
    \end{itemize}
}

{\color{mypurple}\rule{1\linewidth}{1pt}}

\vspace{-0.1cm}
\trapezoidtitleA{项目经历}
\noindent

\parbox{\linewidth}{
    \textbf{LSTM语音识别系统} \\
    \begin{itemize}[leftmargin=*]
        \item 利用pandas和librosa进行数据处理
        \item 实现高效语音识别模型
    \end{itemize}
}

{\color{mypurple}\rule{1\linewidth}{1pt}}

\vspace{-0.1cm}
\trapezoidtitleB{爱好} 
\noindent

\parbox{\linewidth}{
    \faPaintBrush \ 绘画 \ \faCamera \ 摄影 \ \faTableTennis \ 乒乓球 \ \faBadminton \ 羽毛球
}
\end{document}
